\chapter{Analýza požadavků}

Požadavky na aplikaci byly zadány dobře a~přesně, díky tomu, že vzešly přímo od budoucích uživatelů. Aplikace byla nasazena už během vývoje a~průběžně používána, což napomohlo najít nedostatky včas a~uvést ji do stávající podoby.

\section{Přenos velkého objemu dat}

Aplikace musí zvládnout načíst a~zpracovat datastory co největší velikosti. Velikost datastoru se pohybuje v~řádu gigabytů, neuronů mohou být desítky tisíc a~hran desítky milionů. Je potřeba minimalizovat velikost dat pro přenos po síti, a je důležité dát si pozor na nápor na paměť serveru. Výše zmíněná instance měla k dispozici zhruba 8G paměti --- na tento limit se během provozu alespoň ze začátku naráželo často. O něco méně je to problém u klienta. Pochopitelně se zdroji nechceme plýtvat, ale stroje, na kterých se aplikace ve výsledku spouští, patří výzkumníkům a jsou výkonné.

\section{Specifikace složek s datastore}

Datastore k prohlížení se nachází na straně serveru. Aplikaci lze spustit na lokálním počítači, nebo například na serveru se sdílenými adresáři více uživatelů. V obou případech se hodí zadat serveru cestu, odkud může začít procházat adresářový strom. Jednak aby se zrychlilo vyhledávání uživatele, jednak kvůli bezpečnosti v případě veřejně přístupného serveru. Aplikace nesmí akceptovat cestu k datastore, která by nebyla potomkem nakonfigurovaného kořene. Zároveň by mělo být možné specifikovat více kořenů naráz.

\section{Přehledný výpis ADS}

Aplikace je potřeba kvůli zrychlení práce uživatelů. ADS by měly být zobrazeny tak, aby se šlo co nejrychleji zorientovat a najít to, co uživatel chce. ADS mají mnoho stejných parametrů, ty je potřeba umět vyfiltrovat a zobrazit jen ty, ve kterých se liší.

\section{Filtrování a vyhledávání ADS}

Na předchozí požadavek navazuje nutnost mít kvalitní vyhledávací systém. ADS by mělo být možné snadno a intuitivně filtrovat a řadit. Zároveň musí být filtrovací systém dostatečně silný na to, aby zvládal i hodně komplikované dotazy. Filtry by měly být přizpůsobené jednotlivým datovým typům parametrů.

\section{Workspace s ADS}

Načítání konkrétní ADS ze serveru může zabrat netriviální množství času. Je nutné mít možnost některé ADS označit a udržovat je v paměti, aby mezi nimi bylo možné rychle přepínat. Zároveň je nutné mít možnost konkrétní ADS z paměti odstranit a uvolnit zdroje.

\section{Vícenásobné zobrazení}

V určitých případech je rušivé a zdržující muset pořád přepínat mezi několika ADS, obzvlášť pokud je uživatel srovnává a hledá rozdíly a podobnosti. Tehdy je potřeba umět zobrazit více ADS naráz vedle sebe.

\section{Sdílené nastavení}

Každý druh vizualizace má svá specifická nastavení. Někdy se hodí moci upravovat nastavení pro každou ADS zvlášť, ale v případě vícenásobného zobrazení musí existovat možnost upravovat nastavení pro všechny zobrazené vizualizace stejného typu naráz.

\section{Connections}

Aplikace musí umět zobrazit spojení mezi neurony ve stejné vrstvě. Vrstva by měla být vizualizována jako interaktivní scatterplot.

\section{PerNeuronValue}

Aplikace musí umět vizualizovat PerNeuronValue ADS. Vizualizace by měla mít stejnou podobu jako vizualizace Connections, ale neurony by měly mít přiřazenou barvu na základě své hodnoty.

\section{PerNeuronPairValue}

Aplikace musí umět vizualizovat PerNeuronValue ADS. Vizualizace by měla být maticový graf.

\section{AnalogSignalList}

Aplikace musí umět vizualizovat AnalogSignalList ADS.

\section{Udržovatelnost}

Jeden z nejdůležitějších požadavků je udržovatelnost aplikace. Aplikace musí být napsána srozumitelně a rozšiřitelně. Musí být použity moderní technologie, které pravděpodobně brzy nezaniknou. Aplikace musí být dokumentována a důkladně otestována automatickými testy.