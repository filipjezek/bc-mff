
\chapwithtoc{Úvod}

Porozumění lidskému mozku je téma, které v posledním desetiletí nabylo ještě většího významu. Mluví se o něm jak v souvislosti s umělou inteligencí, tak s neuroprotetikami. Klíčovým přístupem k výzkumu se stává výpočetní neurověda --- simulace jednotlivých vrstev neuronů, které napodobují mozkovou strukturu a funkcionalitu. CSNG~(Computational Systems Neuroscience Group)\footnote{\url{https://csng.mff.cuni.cz/}} z Matematicko-fyzikální fakulty za tímto účelem aktivně vyvíjí nástroj Mozaik\footnote{\url{https://github.com/CSNG-MFF/mozaik}}. Mozaik umožňuje specifikovat, spustit a analyzovat simulace neuronových sítí, přičemž veškerá data z takové simulace se ukládají do \emph{datastore}. Během simulace typicky běží několik algoritmů za různých podmínek, každý algoritmus pak generuje jednu nebo více \emph{datových struktur}. Tyto datové struktury je možné načítat pomocí Pythonu a zkoumat např. v Jupyter noteboocích. Dosud ale není možnost nějak uživatelsky přívětivě vizualizovat strukturu neuronových vrstev a spojení, ani uceleně zobrazit a filtrovat všechny datové struktury.

\section*{Cíle}

Cílem této práce je návrh a implementace webové aplikace pro vizualizaci modelu a datových struktur vygenerovaných za běhu Mozaiku. Aplikace by měla poskytnout rozhraní pro dotazy nad metadaty datových struktur, každý druh datové struktury by měl mít vlastní typ vizualizace a vizualizace by měly být interaktivní. Velká část práce s výsledky analýz je komparační, v aplikaci by tedy mělo být snadné porovnávat datové struktury vytvořené s různými parametry.

\section*{Struktura práce}

Nejprve si v kapitole \ref{chap:mozaik} přiblížíme Mozaik. Podíváme se na základy práce s ním, ale hlavně se seznámíme se strukturou datastore, ze kterého budeme získávat data. Také je důležité popsat návaznost na předchozí práce v tomto směru. V kapitole \ref{chap:requirements} formulujeme přesné požadavky na aplikaci. Kapitola \ref{chap:architecture} bude věnována popisu architektury, jejíž implementaci pak rozvedeme v kapitole \ref{chap:implementation}. Speciálně se podíváme na klíčové třídy frontendu v kapitole \ref{chap:frontend}. Posléze si v kapitole \ref{chap:howto} přiblížíme zacházení s programem. Konečně v příloze \ref{app:api} je podrobně popsáno serverové API.
