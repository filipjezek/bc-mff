
\chapwithtoc{Závěr}

Navrhli jsme a implementovali webovou aplikaci pro vizualizaci modelu a datových struktur vygenerovaných za běhu Mozaiku. Splnili jsme všechny vytyčené požadavky.

\begin{description}
  \item[Přenos velkého objemu dat] ze serveru je zajištěný streamováním ve formátu CSV. Je snadné část dat přenášených po síti ve formátu JSON začít také streamovat ve formátu CSV bez změny klienta.
  \item[Specifikace složek s datastore] je možná ve spouštěcím skriptu.
  \item[Přehledný výpis ADS] byl dosažen ve formě stránkované tabulky s buňkami specializovanými na různé datové typy.
  \item[Filtrování a vyhledávání ADS] je možné buď pomocí SQL, nebo pomocí dialogových oken.
  \item[Workspace s ADS] je implementováno ve formě záložek.
  \item[Vícenásobné zobrazení] je podporováno.
  \item[Sdílené nastavení] ADS je rovněž podporováno.
  \item[Connections] jsou vizualizovány jako interaktivní scatterplot.
  \item[PerNeuronValue] je vizualizována jako obarvený interaktivní scatterplot.
  \item[PerNeuronPairValue] je vizualizována jako interaktivní matice.
  \item[AnalogSignalList] je vizualizován jako interaktivní kombinace scatterplotu a čárového grafu.
  \item[Udržovatelnost] byla dosažena volbou frameworku, čistým a přehledným kódem a velkým množstvím automatických testů.
\end{description}

\section*{Možná rozšíření}

\subsection*{Alternativní barevné škály}

V různých případech se hodí použít různé barevné škály, v závislosti na tom, jaké informace jsou pro uživatele důležité a jak s daty chce zacházet\cite{silva2007there}. Zatím neexistuje v aplikaci možnost vybírat z jiných škál než z periodické a neperiodické. Tato funkčnost by přinesla možnost efektivnější analýzy.

\subsection*{Permutace maticové vizualizace}

Současná maticová vizualizace je užitečná pro rychlé zjištění specifické hodnoty. Je ale obtížné z ní vypozorovat existující trendy. Pro takovou funkčnost je nutné matici vhodně uspořádat\cite{chen2007handbook}. Bylo by vhodné implementovat jeden nebo více existujících algoritmů, například eliptickou seriaci\cite{chen2002generalized}